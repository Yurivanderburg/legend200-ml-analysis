Neutrinoless double beta decay is a hypothesized nuclear process whose observation would have far-reaching implications for particle physics, providing insight into the fundamental nature of neutrinos and the possible violation of lepton number conservation. 
The Large Enriched Germanium Experiment for Neutrinoless double beta Decay (LEGEND) experiment aims to detect this rare decay using high-purity germanium detectors enriched in $^{76}$Ge, where pulse shape discrimination plays a critical role in suppressing background events. 

This thesis presents the development and application of a novel pulse shape discrimination algorithm using a Transformer-based neural network trained on time-series charge waveforms. 
The pulse shape discrimination efficiency at the energy corresponding to the neutrinoless double beta decay signal was evaluated using data from Inverted Coaxial Point Contact detectors produced by Mirion Technologies. The Transformer-based classifier achieved an efficiency of $(86.7 \pm 1.3)$\%, which is consistent with the $(84.3 \pm 0.5)$\% obtained using the conventional A/E method, where A/E is the ratio of the maximum current amplitude to the reconstructed energy. A two-sided $p$-value of $p \approx 0.08$ indicates no significant difference between the two approaches at the 5\% level. The total detection efficiency achieved with the Transformer-based approach is $(66.8 \pm 2.1)$\%, compared to $(65.0 \pm 2.0)$\% for the A/E method. 
A Bayesian statistical framework is employed to propagate uncertainties in the efficiency and evaluate the resulting impact on LEGEND's neutrinoless double beta decay discovery sensitivity. 

If LEGEND-200 -- the first of two stages of the LEGEND experiment, with a total target mass of 200~kg -- reaches its design goals of a background index of $2 \times 10^{-4}$~counts/(keV$\cdot$kg$\cdot$yr) and a total exposure of $1000 \; \mathrm{kg} \cdot \mathrm{yr}$, the projected sensitivity to the neutrinoless double beta decay half-life is $T^{0 \nu}_{1/2} = 1.22^{+0.33}_{-0.37} \times 10^{27}$~yr for A/E and $T^{0 \nu}_{1/2} = 1.25^{+0.34}_{-0.37} \times 10^{27}$~yr for the Transformer-based method. 

While the Transformer-based classification does not yet significantly enhance the experimental sensitivity, it demonstrates the potential of advanced machine learning techniques in improving background rejection and sensitivity. Further development and validation, such as through refined data cleaning procedures, better balanced training sets, and enhanced detector-specific modeling, are required to consistently outperform conventional methods. 
Continued progress along these lines could pave the way for the robust integration of machine learning models into the LEGEND analysis workflow, ultimately helping to maximize discovery sensitivity.